\newglossaryentry{nfc}{
name=NFC,
%plural=dispositivos inteligentes,
%first={},
description={--}
}\newglossaryentry{RFID}{
name=RFID,
%plural=dispositivos inteligentes,
%first={},
description={--}
}

\newglossaryentry{bluetooth}{
name=bluetooth,
plural=bluetooth,
%first={},
description={Bluetooth}
}
\newglossaryentry{board}{
name=board,
plural=boards,
description={En el ámbito de la computación ubícua, dispositivo electrónico de grandes dimensiones interactuable por varias personas}
}\newglossaryentry{computacionUbicua}{
name=computación ubícua,
%plural=dispositivos inteligentes,
%first={},
description={Ver smartdevices}
}\newglossaryentry{dispositivointeligente}{
name=dispositivo inteligente,
plural=dispositivos inteligentes,
%first={},
description={Ver smartdevices}
}\newglossaryentry{iot}{
symbol=IoT,
name=Internet of Things,
plural=Internet de las Cosas,
%first={Internet of Things},
description={Paradigma basado en el que las cosas o \texttt{things} pueden comunicarse y colaborar entre ellas mediante su conexión a un red}
}\newglossaryentry{monitorización}{
name=monitorización,
plural=monitorización,
%first={},
description={monitorización}
}
\newglossaryentry{movilidad}{
name=movilidad,
plural=movilidad,
%first={},
description={movilidad}
}\newglossaryentry{nodo}{
name=Nodo de monitorización,
%plural=dispositivos inteligentes,
%first={},
description={--}
}\newglossaryentry{pad}{
%symbol=PA,
name=pab,
plural=pabs,
description={En el ámbito de la computación ubícua, dispositivo electrónico del tamaño de la palma de la mano}
}\newglossaryentry{raspberrypi}{
name=Raspberry Pi,
%plural=dispositivos inteligentes,
%first={},
description={--}
}\newglossaryentry{raspbian}{
name=Raspbian,
%plural=dispositivos inteligentes,
%first={},
description={--}
}\newglossaryentry{sbc}{
name=sbc,
%plural=dispositivos inteligentes,
%first={},
description={--}
}\newglossaryentry{smartcar}{
name=smartcar,
plural=smartcars,
%first={},
description={Automóvil inteligente}
}\newglossaryentry{smartcity}{
name=\texttt{smart city},
plural=\texttt{smart cities},
%first={},
description={Ciudad inteligente, ver Sección \ref{sec:c02:smartcities} para una descripción formal}
}\newglossaryentry{smartdevice}{
name=smartdevice,
plural=smartdevices,
%first={},
description={smartdevice}
}\newglossaryentry{smartphone}{
name=smartphone,
plural=smartphones,
%first={},
description={Teléfono móvil inteligente}
}
\newglossaryentry{smartwatch}{
name=smartwatch,
plural=smartwatches,
%first={},
description={Reloj inteligente}
}\newglossaryentry{softcomputing}{
name=softcomputing,
plural=softcomputing,
%first={},
description={softcomputing}
}\newglossaryentry{tab}{
%symbol=tab,
name=tab,
plural=tabs,
description={En el ámbito de la computación ubícua, dispositivo electrónico de escasos centímetros que puede ser llevado encima o vestido}
}\newglossaryentry{tic}{
name=TIC,
plural=TICs,
%first={},
description={Tecnologías de la Información y la Comunicación}
}\newglossaryentry{wifi}{
name=wifi,
plural=wifi,
%first={},
description={wifi}
}
\newglossaryentry{WLAN}{
name=WWAN,
%plural=dispositivos inteligentes,
%first={},
description={Una red de área local inalámbrica, también conocida como WLAN (del inglés wireless local area network), es un sistema de comunicación inalámbrico para minimizar las conexiones cableadas.}
}

\newglossaryentry{WPAN}{
name=WPAN,
%plural=dispositivos inteligentes,
%first={},
description={Personal Area Network (PAN), Red de Área Personal, es una red de computadoras para la comunicación entre distintos dispositivos cercanos al punto de acceso.}
}

\newglossaryentry{WWAN}{
name=WWAN,
%plural=dispositivos inteligentes,
%first={},
description={--}
}

