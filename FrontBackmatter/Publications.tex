%*******************************************************
% Publications
%*******************************************************
\pdfbookmark[1]{Publicaciones}{publicaciones}
\chapter*{Publicaciones}

%\graffito{This is just an early --~and currently ugly~-- test!}
%This might come in handy for PhD theses: some ideas and figures have appeared previously in the following publications:

%\noindent Put your publications from the thesis here. The packages \texttt{multibib} or \texttt{bibtopic} etc. can be used to handle multiple different bibliographies in your document.

Este trabajo de tesis se avala con los siguientes artículos en revista:

\begin{refsection}[jcr]
    \DeclareFieldFormat{labelnumberwidth}{}
    \setlength{\biblabelsep}{0pt}
    \small
    \nocite{*} % is local to to the enclosing refsection
    \printbibliography[heading=none]
\end{refsection}

Adicionalmente, se ha publicado un capítulo de libro:

\begin{refsection}[libros]
    \DeclareFieldFormat{labelnumberwidth}{}
    \setlength{\biblabelsep}{0pt}
    \small
    \nocite{*} % is local to to the enclosing refsection
    \printbibliography[heading=none]
\end{refsection}

Las siguientes participaciones en congresos:

\begin{refsection}[congresos]
    \DeclareFieldFormat{labelnumberwidth}{}
    \setlength{\biblabelsep}{0pt}
    \small
    \nocite{*} % is local to to the enclosing refsection
    \printbibliography[heading=none]
\end{refsection}

Aunque no estén directamente relacionados con la temática de la tesis, las técnicas y procedimientos aprendidos han sido también empleados en artículos de otras líneas de investigación del autor:

\begin{refsection}[jcrNO]
    \DeclareFieldFormat{labelnumberwidth}{}
    \setlength{\biblabelsep}{0pt}
    \small
    \nocite{*} % is local to to the enclosing refsection
    \printbibliography[heading=none]
\end{refsection}

Como colaboración en artículos de revista de otros investigadores\footnote{La inclusión de estos artículos es únicamente a nivel curricular. Ninguno de las partes de dichos artículos son empleados en ninguna de las partes de esta tesis Doctoral.}:

\begin{refsection}[jcrNONO]
    \DeclareFieldFormat{labelnumberwidth}{}
    \setlength{\biblabelsep}{0pt}
    \small
    \nocite{*} % is local to to the enclosing refsection
    \printbibliography[heading=none]
\end{refsection}

Adicionalmente, en otras líneas de investigación no directamente relacionadas con la tesis, se han publicado los siguientes trabajos en congresos:

\begin{refsection}[congresosNO]
    \DeclareFieldFormat{labelnumberwidth}{}
    \setlength{\biblabelsep}{0pt}
    \small
    \nocite{*} % is local to to the enclosing refsection
    \printbibliography[heading=none]
\end{refsection}

%\emph{Attention}: This requires a separate run of \texttt{bibtex} for your \texttt{refsection}, \eg, \texttt{ClassicThesis1-blx} for this file. You might also use \texttt{biber} as the backend for \texttt{biblatex}. See also \url{http://tex.stackexchange.com/questions/128196/problem-with-refsection}.